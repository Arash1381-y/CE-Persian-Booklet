\begin{flushright}
    در ابتدا فرض کنید که سیستم ما شامل یک روت دایرکتوری می‌باشد که در جایی از حافظه اطلاعاتی از آن ذخیره شده است.
    روت دایکتوری می‌تواند چند کلاستر باشد.
    فرض می‌کنیم کلاستر آغازی این دایرکتوری، کلاستر ۰ است.
    حال سوال این است که کلاستر دوم کجاست؟ و باقی اطلاعات دایرکتوری در کجا ذخیره شده است.
    از کجا می‌شود فهمید که کلاستر بعدی کجاست؟


    در فایل سیستم FAT از روشی مشابه با linked list که در جلسه گذشته به آن اشاره شد، استفاده می‌شود.
    به عبارتی هر کلاستر که شامل اطلاعاتی از فایل/دایرکتوری است، یک اشاره‌گر نیز دارد که به کلاستر بعدی که شامل باقی اطلاعات است اشاره می‌کند.
    در صورتی که باقی اطلاعات در کلاستر دیگری وجود نداشت، می‌توان به صورت قرار دادی یک مقدار خاص را در آن قرار داد.(به طور مثال مقداری بیشتر از آخرین آدرس کلاستر موجود)


    برای آشنایی با شیوه ذخیره‌سازی دایرکتوری‌ها ابتدا بررسی می‌کنیم که یک دایرکتوری شامل چه اطلاعاتی می‌باشد.
    هر دایرکتوری شامل اطلاعاتی چون
    \begin{enumerate}
        \item اندازه دایرکتوری
        \item نام دایرکتوری
        \item تاریخ ایجاد
        \item محتوای دایرکتوری
        \item صاحب دایرکتوری
    \end{enumerate}
    این اطلاعات در کلاستر‌های اول ذخیره می‌شود.
    حال سوال این است که ما directory root را فقط خواندیم.
    باقی فایل‌ها را چطور بخوانیم؟ از آنجا که داخل هر directory اطلاعات مربوط به entry ها را داشتیم (که در واقع آدرس کلاستر اول هر entry است)،
    چطور بین دیتاها در حافظه تمایز قائل می‌شویم؟برای رفع این مشکل یک ساختار برای ذخیره اطلاعات در نظر میگیریم.
    یک کلاستر یا کلا برای یک فایل است یا خالی است.
    سایز خود فایل در متادیتای آن نوشته شده است.
    با استفاده از این می‌توانیم بفهمیم که باید تا کجای کلاستر را بخوانیم. (البته می‌شود آخرش هم EOF گذاشت).

     اگر همه چیز را انتزاعی ببینیم می‌توانیم یک فایل سیستم را در نظر بگیریم که با یک ساختار سلسله مراتبی، به صورت درختی، در آن اطلاعات به صورت فایل ذخیره می‌شود.
     تا اینجا می‌توانیم با استفاده از این انتزاع فایل، اطلاعات خود را ذخیره و بازیابی کنیم.
     اما سوال این است که آیا انتزاعی فراتر از فایل هم وجود دارد؟

\end{flushright}