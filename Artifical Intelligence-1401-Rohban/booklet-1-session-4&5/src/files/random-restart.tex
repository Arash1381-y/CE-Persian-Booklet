در این روش یک محدوده‌زمانی برای یافتن جواب برای مسئله تعیین می‌کنیم و اگر در زمان تعیین شده، راه‌حل یافت نشد به صورت رندوم به یک state می‌رویم.

\hfill\break
فرض کنید می‌خواهیم از این روش برای حل مسئله 8 وزیر استفاده کنیم.
امیدریاضی تعداد restart ها چه عددی است؟

\begin{displaymath}
    E[n] = \sum_{i}{P}(i) \times R(i)
\end{displaymath}
که در آن $R(i)$ تعداد restart های انجام شده در روش i ام می‌باشد.
همچنین $P(i)$ احتمال رخداد روش i ام می‌باشد.
\hfill\break
در قسمت hill-climbing دیدم که احتمال موفقیت برابر 14 \% می‌باشد لذا داریم:
\begin{displaymath}
    E[n] = 0.14 \times 0 + 0.86 \times 0.14 \times 1 + 0.86_2 \times 0.14 \times 2 + \dots
\end{displaymath}
همانطور که میبینید تعداد restart ها از یک توزیع هندسی پیروی می‌کند و از آنجا که امیدریاضی توزیع هندسی برابر است با $1/p$ داریم:
\begin{displaymath}
    E[n] = 1/0.14 = 7.14
\end{displaymath}
\hfill\break
با توجه به تعداد متوسط restart ها
گام‌های لازم برای رسید به جواب چگونه محاسبه می‌شود؟
برای محاسبه کافی است از امیدریاضی شرطی حساب کنیم.
لذا داریم:

\begin{displaymath}
    E[steps] = E_{R}(E(steps|R)) = E_{R}(3\times(R - 1) + 4) \cong 22
\end{displaymath}