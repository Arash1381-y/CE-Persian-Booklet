در بخش قبل دیدیم که یکی از چالش‌هایی که در روش hill-climbing با آن روبرو بودیم برگشتن به state هایی بود که قبلا بررسی شده‌اند.
در روش Tabu search، یک صف با سایز k داریم و پس از مشاهده هر state آن را به صف وارد می‌کنیم و در صورت پر بودن صف، آخرین state را خارج می‌نماییم.
نکته قابل توجه آن است که نمی‌توانیم به state هایی برویم که درون صف وجود دارند و اینکار باعث می‌شود تا مشاهده state های بررسی شده تاحد خوبی کاهش یابد.

